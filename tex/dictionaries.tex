

\section{Dictionaries}
\scalebox{0.8}{\textit{Reference: Gaddis Chapter 9}}
\smallskip

Dictionaries allow us to store key-value pairs. It's kind of like having one row in a table. Keys are like column names and values are the row-column cell value. Key-value pairs are created as \code{key: value}, then separated by commas and wrapped in curly braces, \code{\{\}}, to create a dictionary. Consider the example below.

\begin{lstlisting}[language = Python]
workout = {'user': 'Velma', 'fitness_discipline': 'cycling', 'instructor': 'Matt Wilpers'}
\end{lstlisting}

A specific value is access by indexing the dictionary by the key. 

\begin{lstlisting}[language = Python]
print(workout['user'])
\end{lstlisting}

\smallskip
We can add new key-value pairs by assigning the value to the dictionary at that key. 

\begin{lstlisting}[language = Python]
# Build a dictionary from scratch
journal = dict() # creates an empty dictionary, can also use {}

journal['2020-10-03'] = "Today I learned a lot of Python. It was buckets of fun."
\end{lstlisting}

\smallskip
\noindent The \code{in} operator works on dictionaries by searching the keys. 

\begin{lstlisting}[language = Python]
# Help translate bad journalism
media_translator = {'is caused by': 'is correlated with'}

print('is caused by' in media_translator)
print('is correlated with' in media_translator)
\end{lstlisting}


You can access the keys with the \code{.keys()} operator and values with the \code{.values()} operator. So \code{x in some_dict} is actually a shorthand for \code{x in some_dict.keys()}.

\smallskip

\noindent Dictionary keys must be immutable. Tuples are fine. 

\begin{lstlisting}[language = Python]
# Economist Santa

gifts = {} # could also use dict() here 

for child in ['Anna', 'Boris']:
    for year in [2020, 2021]:
        
        key = child, year # this is a tuple just like (child, year)
        
        gifts[key] = 'money'
        
print(gifts)
\end{lstlisting}

